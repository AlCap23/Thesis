%
%		Header-Datei
%

% Nummerierungstiefe von section/subsection/subsubsection
\setcounter{secnumdepth}{3} 


%Absatzeinzug einstellen
\setlength{\parindent}{1em}
\setlength{\parskip}{0pt}

% Deutsche Rechtschreibung und deutscher Zeichensatz
%\usepackage{german,ngerman}
%\usepackage[ansinew]{inputenc}  

% Aktuelles Datum ermitteln
\usepackage[ngerman]{datenumber}

% Erweiterung f�r ein deutsches Literaturverzeichnis
\bibliographystyle{abbrvdin}

% Erweiterte Mathematikbibliotheken
\usepackage{amsmath}
\usepackage{amssymb}

% Automatisches Formelzeichenverzeichnis
\usepackage{nomencl}
\makenomenclature
\renewcommand{\nomname}{Nomenclature}
\renewcommand{\nompreamble}{The list of symbols used}
\usepackage{etoolbox}
\renewcommand\nomgroup[1]{%
  \item[\bfseries
  \ifstrequal{#1}{G}{Greek Letters}{%
  \ifstrequal{#1}{R}{Roman Letters}{%
  \ifstrequal{#1}{O}{Other Symbols}{}}}%
]}

%Vectoren und Matrizen
\usepackage[normalem]{ulem}
\newcommand{\ve}[1]{\mbox{\boldmath$#1$}}
\newcommand{\ma}[1]{\mbox{\boldmath$#1$}}
\newcommand{\eR}{\mbox{$\varepsilon I\;R$}}

% Lange Tabellen
\usepackage{longtable}

% Ma�einhaeiten-Darstellung verbessern
\usepackage{units}

% Extra Symbole f�r die Verwendung im Text (z.B. \textcelsius)
%(werden nicht aus der Nexus-Schriftart einbunden)
\usepackage{textcomp}

% Einbinden von externen PDF Dateien
\usepackage[final]{pdfpages}

% Zum einbinden von Grafiken  
\usepackage{graphicx,import}
\usepackage{subcaption}  %erlaubt das erstellen von subfigures
\usepackage{float} %erm�glicht das erstellen von "non-floating figures"
% Setzen des Verzeichnisses für Grafiken
\graphicspath{{./Graphics/}}

%Hyperref f�r Verlinkungen im Dokument und URL's
%hyphens f�r verbesserten Zeilenumbruch in url's
\PassOptionsToPackage{hyphens}{url}
\usepackage[	breaklinks={true},
 						colorlinks	={true},
 					  	linkcolor	={tubsBlack},
  						citecolor	={tubsBlack},
  						filecolor	={tubsBlack},
  						urlcolor	={tuSecondaryMedium100},						
 					]{hyperref} %Interne Links in Schwarz und externe URL's in bg Farbe
 					
%Zus�tzliche Buchstaben und Zahlen erlauben f�r Zeilenumbruch bei url's (verbessert den Blocksatz)				
\expandafter\def\expandafter\UrlBreaks\expandafter{\UrlBreaks%  save the current one
  \do\a\do\b\do\c\do\d\do\e\do\f\do\g\do\i\do\j%
  \do\k\do\l\do\m\do\n\do\o\do\q\do\r\do\s%
  \do\u\do\v\do\w\do\x\do\y\do\z\do\A\do\B\do\C\do\D%
  \do\E\do\F\do\G\do\H\do\I\do\J\do\K\do\L\do\M\do\N%
  \do\O\do\P\do\Q\do\R\do\S\do\T\do\U\do\V\do\W\do\X%
  \do\Y\do\Z\do\1\do\2\do\3\do\4\do\5\do\6\do\7\do\8\do\9} 	
  
% R�ckseiten-Elemente
\address{% Addresse
  Technische Universit�t Braunschweig\\
  Institut f�r Thermodynamik\\
  Hans-Sommer-Strasse 5\\
  38106 Braunschweig}
\backpageinfo{}				

%Makro zur Formatierung des Anhangs
\newcommand{\Anhang}{
\appendix
\chapter*{Anhang}
\addcontentsline{toc}{chapter}{Anhang}
\setcounter{chapter}{1}
\markboth{Anhang}{Anhang}
\label{Anhang}}

%Makro zur Formatierung von der Titelseite bis zum Inhaltsverzeichniss
\newcommand{\Anfang}{
\pagenumbering{Roman}
\begin{titlepage}
%Designhelper f�r Erstellung 
%\showdesignhelper

\showtubslogo[right]
\showlogo{\includegraphics{ift_4C.pdf}}
\TOPLINE

%Titelbild f�r die Arbeit
\begin{titlerow}[bgimage=\TITELBILD ,imagefit=cropped]{3}


\hfill\vfill
\end{titlerow}
\begin{titlerow}[bgcolor=tuSecondaryDark100, c]{2}

\begin{fitbox}{\textwidth}{100pt}
\raggedright
\color{tubsWhite}
\textbf{\TITLE}
\end{fitbox}
\addvspace{1,5em}
\begin{Large}
\raggedright
\color{tubsWhite}
\textbf{\TYPE}\vfill
\end{Large}

\end{titlerow}
\begin{titlerow}[bgcolor=tuSecondaryMedium80]{3}
\color{tubsWhite}


\begin{tabbing}
\hspace*{2.0cm}   \=  \hspace*{12.0cm}        \= \kill
\textbf{Autor:} \> \textbf{\AUTHOR} \\
\textbf{Betreuer:} \> \textbf{\BETREUER}\\
\textbf{Pr�fer:} \> \textbf{Prof. Dr.-Ing. J�rgen K�hler}\\


 \> \textbf{\today}
\end{tabbing}
\hfill\vfill
\end{titlerow}
\hfill\vfill
\end{titlepage}

\includepdf[pages=-]{\AUFGABENSTELLUNG}
\addchap*{Eidesstattliche Erkl\"arung}


\vspace*{5cm}


Hiermit erkl\"are ich eidesstattlich, dass ich diese Arbeit eigenst\"andig angefertigt und keine anderen als die angegebenen Hilfsmittel verwendet habe.
\vspace{3cm}\\
Braunschweig den \today
\tableofcontents
\listoffigures
\makenomenclature
\printnomenclature
\pagenumbering{arabic}
\setcounter{page}{1}}



%Makro zum einpassen des Titels in die Titelbox
\usepackage{environ}
\newdimen\fontdim
\newdimen\upperfontdim
\newdimen\lowerfontdim
\newif\ifmoreiterations
\fontdim12pt
\makeatletter
\NewEnviron{fitbox}[2]{% \begin{fitbox}{<width>}{<height>} stuff \end{fitbox}
  \def\buildbox{%
    \setbox0\vbox{\hbox{\minipage{#1}%
      \fontsize{\fontdim}{1.2\fontdim}%
      \selectfont%
      \stuff%
    \endminipage}}%
    \dimen@\ht0
    \advance\dimen@\dp0
  }
  \def\stuff{\BODY}% Store environment body
  \buildbox
  % Compute upper and lower bounds
  \ifdim\dimen@>#2
    \loop
      \fontdim.5\fontdim % Reduce font size by half
      \buildbox
    \ifdim\dimen@>#2 \repeat
    \lowerfontdim\fontdim
    \upperfontdim2\fontdim
    \fontdim1.5\fontdim
  \else
    \loop
      \fontdim2\fontdim % Double font size
      \buildbox
    \ifdim\dimen@<#2 \repeat
    \upperfontdim\fontdim
    \lowerfontdim.5\fontdim
    \fontdim.75\fontdim
  \fi
  % Now try to find the optimum size
  \loop
    %\message{Bounds: \the\lowerfontdim\space
    %         \the\fontdim\space \the\upperfontdim^^J}
    \buildbox
    \ifdim\dimen@>#2
      \moreiterationstrue
      \upperfontdim\fontdim
      \advance\fontdim\lowerfontdim
      \fontdim.5\fontdim
    \else
      \advance\dimen@-#2
      \ifdim\dimen@<10pt
        \lowerfontdim\fontdim
        \advance\fontdim\upperfontdim
        \fontdim.5\fontdim
        \dimen@\upperfontdim
        \advance\dimen@-\lowerfontdim
        \ifdim\dimen@<.2pt
          \moreiterationsfalse
        \else
          \moreiterationstrue
        \fi
      \else
        \moreiterationsfalse
      \fi
    \fi
  \ifmoreiterations \repeat
  \box0% Typeset content
}
\makeatother