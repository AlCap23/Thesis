\chapter{Application to a Two Input Two Output First Order Time Delay System}

\section{General Example for First Order Time Delay Process Models}
\label{c:control:s:example}

As an example a system with two inputs and two outputs is investigated. Every transfer function is modelled as an FOTD.

\begin{align}
\begin{split}
\ma{G} = \begin{bmatrix}
g_{11} & g_{12} \\
g_{21} & g_{22}
\end{bmatrix} & ~,g_{ij} = \frac{K_{ij}}{T_{ij} s +1 } e^{-L_{ij}s}
\end{split}
\label{c:control:e:examplesys}
\end{align}

For the system described in Eq. \ref{c:control:e:examplesys} a controller is now designed via the Algorithm proposed by Aström et.al..\\

At first the decoupler is designed via the inverse static gain of the system:


\begin{align}
\begin{split}
\ma{D} &= \ma{G}|_{s=0}^{-1}\\
&= \frac{1}{K_{11}K_{22}-K_{12}K_{21}} 
\begin{bmatrix}
K_{22} & -K_{21} \\
-K_{12} & K_{11}
\end{bmatrix}
\end{split}
\label{c:control:e:exampleDecoupler}
\end{align}

The transformed system $\ma{G}^*$ is given by:

\begin{align}
\begin{split}
\ma{G}^* &= \ma{G}\ma{D}\\
&= \frac{1}{K_{11}K_{22}-K_{12}K_{21}} 
\begin{bmatrix}
g_{11} & g_{12} \\
g_{21} & g_{22}
\end{bmatrix}
\begin{bmatrix}
K_{22} & -K_{21} \\
-K_{12} & K_{11}
\end{bmatrix} \\
&= \frac{1}{K_{11}K_{22}-K_{12}K_{21}}
\begin{bmatrix}
K_{22}g_{11}-K_{12}g_{12} & -K_{21}g_{11}+K_{22}g_{12} \\
K_{22}g_{21}-K_{12}g_{22} &
-K_{21}g_{21}+K_{11}g_{22}
\end{bmatrix}
\end{split}
\label{c:control:e:exampletransformed}
\end{align}

From Eq. \ref{c:control:e:exampletransformed} it is clear that the entries $g_{ij}^*$ are linear combinations of FOTD transfer functions. Due to the properties of the exponential function the superposition principle does not hold. Hence a controller via the AMIGO algorithm can only be designed if a sufficient approximation of the linear combination as a FOTD can be formulated:

\begin{align}
\begin{split}
g_{ij}^* &\approx \frac{K_{ij}^*}{T_{ij}^*s+1}e^{-L_{ij}^* s} 
\end{split}
\label{c:control:e:exampleFOTDapprox}
\end{align}

Assuming a sufficient approximation can be found and the resulting error is minimal the diagonal controller can be designed. A choice for a PI-Structure with set point weighting $b=0$ holds:

\begin{align}
\begin{split}
\ma{K}_y^* &= \begin{bmatrix}
-K_{P1}^* - K_{I1}^*\frac{1}{s} & 0 \\
0 & -K_{P2}^* - K_{I2}^*\frac{1}{s}
\end{bmatrix} \\
\ma{K}_r^* &= \begin{bmatrix}
K_{I1}^*\frac{1}{s} & 0 \\
0 & K_{I2}^*\frac{1}{s}
\end{bmatrix} 
\end{split}
\label{c:control:e:exampleTransformedControler}
\end{align}

With parameters $K_{P,i},K_{I,i} \in \mathbb{R}$ are calculated via the AMIGO Tuning Rules as given earlier. Since the approximation given in Eq.\ref{c:control:e:exampleFOTDapprox} holds an inevitable error so do the parameter. \\

The coupling matrix of the antidiagonal parts is given by:

\begin{align}
\begin{split}
\ma{\Gamma}_A &= \left[\frac{d}{ds}\ma{G}|^*_{s=0}\right]_A s \\
&= \begin{bmatrix}
0 & \frac{-K_{21}K_{11}(T_{11}-L_{11}) + K_{22}K_{12}(T_{12}-L{12})}{K_{11}K_{22}-K_{12}K_{21}} \\
\frac{-K_{12}K_{22}(T_{22}-L_{22}) + K_{22}K_{21}(T_{21}-L_{21})}{K_{11}K_{22}-K_{12}K_{21}} & 0
\end{bmatrix} s\\
&\approx \begin{bmatrix}
0 & K_{12}^* (T_{12}^* - L_{12}^*) \\
K_{21}^* (T_{21}^* - L_{21}^*) & 0 
\end{bmatrix} s
\end{split}
\label{c:control:e:exampleTransformedCoupling}
\end{align}

From Eq. \ref{c:control:e:exampleTransformedCoupling} the dependency of the coupling on the both the static gain of the system and the dynamical behaviour can be observed. This coincides with the statements of LUNZE ZITIEREN.\\

Detuning the controller requires to define both  maximum allowed interactions $h_{ij,Max}$ and maximum sensitivity $M_{S,i}$ of the closed loop:

\begin{align*}
\begin{split}
\ma{H}_{A,Max} &= \begin{bmatrix}
0 & h_{12,Max} \\
h_{21,Max} & 0 
\end{bmatrix}\\
\ma{M}_{S} &= \begin{bmatrix}
M_{S,1} & 0 \\
0 & M_{S,2}
\end{bmatrix}
\end{split}
\end{align*}

Solving Eq. \ref{c:control:e:Hmax*} for the set point weighting controller holds:

\begin{align}
\begin{split}
\ma{K}^*_r &\leq \ma{\Gamma}^{-*}_{A,Max}\ma{M}_S^{-1}\ma{H}^*_{A,Max} \\ 
&\leq \frac{1}{\det(\ma{M}_S)}\ma{\Gamma}^{-*}_{A,Max}\ma{H}^*_{A,Max}\\
&\leq \frac{1}{M_{S,1}M_{S,2}}
\begin{bmatrix}
\frac{h_{21,Max}}{K_{12}^* (T_{12}^* - L_{12}^*)} & 0\\
0 & \frac{h_{12,Max}}{K_{21}^* (T_{21}^* - L_{21}^*)} 
\end{bmatrix}
\end{split}
\end{align}

Now the modified Algorithm proposed in this thesis is applied to the same System. First, we design the controller as a function of the main diagonal entries $g_{ii}$ once again using the AMIGO Tuning rules:

\begin{align}
\begin{split}
\ma{K}_y &= \begin{bmatrix}
-K_{P1} - K_{I1}\frac{1}{s} & 0 \\
0 & -K_{P2} - K_{I2}\frac{1}{s}
\end{bmatrix} \\
\ma{K}_r &= \begin{bmatrix}
K_{I1}\frac{1}{s} & 0 \\
0 & K_{I2}\frac{1}{s}
\end{bmatrix} 
\end{split}
\label{c:control:e:exampleControler}
\end{align}

The splitter $\ma{\Sigma}$ is likewise designed by the steady state of the system as:

\begin{align}
\begin{split}
\ma{\Sigma} &= \ma{D}_A\ma{D}_D^{-1} \\
& = \begin{bmatrix}
0 & -\frac{K_{12}}{K_{11}} \\
-\frac{K_{21}}{K_{22}} & 0
\end{bmatrix}
\end{split}
\end{align}

To test for interaction define the maximum interaction and the sensitivity like in Eq. FEHLT. The anti diagonal parts of the Taylor series can be identified as:

\begin{align}
\begin{split}
\ma{\Gamma}_A &= \frac{d}{ds}\left[\ma{G}_A + \ma{G}_D\ma{\Sigma}\right]|_{s=0} \\
&= \begin{bmatrix}
0 & K_{12}(T_{12}-L_{12}) - K_{11}\frac{K_{12}}{K_{11}}(T_{11}-L_{11}) \\
K_{21}(T_{21}-L_{21}) - K_{22}\frac{K_{21}}{K_{22}}(T_{22}-L_{22}) & 0 
\end{bmatrix} \\
&= \begin{bmatrix}
0 & K_{12}(T_{12}-L_{12} - T_{11}+L_{11}) \\
K_{21}(T_{21}-L_{21} - T_{22}+L_{22}) & 0 
\end{bmatrix} 
\end{split}
\end{align}

To detune the controller solving Eq. \ref{c:control:e:Hmax*} for the integral controller as before holds:

\begin{align}
\begin{split}
\ma{K}_I &\leq \ma{\Gamma}_A^{-1} \ma{M}_S^{-1} \ma{H}_{A,Max} \\
&\leq \frac{1}{\det(\ma{M}_S)} \ma{\Gamma}_A^{-1} \ma{H}_{A,Max} \\
&\leq \frac{1}{M_{S,1}M_{S,2}}\begin{bmatrix}
\frac{h_{12,Max}}{K_{12}(T_{12}-L_{12}-T_{11}+L_{11})} & 0 \\
0 &\frac{h_{21,Max}}{K_{21}(T_{21}-L_{21}-T_{22}+L_{22})}
\end{bmatrix}
\end{split}
\end{align}
\subsection*{Overview}

Three Methods are presented above. 

\begin{figure}[H]
\begin{minipage}[b]{\textwidth}
\centering
\includegraphics[width = \textwidth]{newplot__5_.png}
\caption{Different Methods for MIMO Controller Design}
\label{c:control:f:MIMODesign}
\end{minipage}
\end{figure}
\begin{figure}[H]
\begin{minipage}[b]{\textwidth}
\centering
\includegraphics[width = \textwidth]{newplot__6_.png}
\caption{Different Methods for MIMO Controller Design}
\label{c:control:f:MIMODesign2}
\end{minipage}
\end{figure}


