\chapter{Einleitung}
\label{sec:Einleitung}
Hier stehen einige einf�hrende Worte in das Thema der Arbeit.



\section{Zielsetzung und Gliederung}
\label{sec:Zielsetzung}
Was ist Sinn und Zweck der Arbeit, wie ist sie aufgebaut und welche Themen werden behandelt \cite{koehler}.


\section{Geschichte/Literatur}
\label{sec:Geschichte_Literatur}
Optionale Betrachtung der Thematik wie sie in der Geschichte und/oder Literatur Erw�hnung findet. Hier sollten auch schon der eine Literaturverweise auftauchen.

\section{Formatierung}
\label{sec:Formatierung}
Um weitere Informationen zur Formatierung dieser Dokumente zu erhalten, schauen sie in der Anleitung der TU Braunschweig zum Corporate Design mit \LaTeX{} nach.\cite{tubsLatex} \par
Mit dem gew�hlten Zitierstill lassen sich auch Patente\cite{Patent}, sowie Internetseiten\cite{Website} und Normen. Alle Zitate erfolgen nach \cite{Norm}. 
