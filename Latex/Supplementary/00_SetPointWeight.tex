% --------------------------------------------------------------
% This is all preamble stuff that you don't have to worry about.
% Head down to where it says "Start here"
% --------------------------------------------------------------
 
\documentclass[12pt]{article}
 
\usepackage[margin=1in]{geometry} 
\usepackage{amsmath,amsthm,amssymb}
 
\newcommand{\N}{\mathbb{N}}
\newcommand{\Z}{\mathbb{Z}}
 
\newenvironment{theorem}[2][Theorem]{\begin{trivlist}
\item[\hskip \labelsep {\bfseries #1}\hskip \labelsep {\bfseries #2.}]}{\end{trivlist}}
\newenvironment{lemma}[2][Lemma]{\begin{trivlist}
\item[\hskip \labelsep {\bfseries #1}\hskip \labelsep {\bfseries #2.}]}{\end{trivlist}}
\newenvironment{exercise}[2][Exercise]{\begin{trivlist}
\item[\hskip \labelsep {\bfseries #1}\hskip \labelsep {\bfseries #2.}]}{\end{trivlist}}
\newenvironment{reflection}[2][Reflection]{\begin{trivlist}
\item[\hskip \labelsep {\bfseries #1}\hskip \labelsep {\bfseries #2.}]}{\end{trivlist}}
\newenvironment{proposition}[2][Proposition]{\begin{trivlist}
\item[\hskip \labelsep {\bfseries #1}\hskip \labelsep {\bfseries #2.}]}{\end{trivlist}}
\newenvironment{corollary}[2][Corollary]{\begin{trivlist}
\item[\hskip \labelsep {\bfseries #1}\hskip \labelsep {\bfseries #2.}]}{\end{trivlist}}
 
\begin{document}
 
% --------------------------------------------------------------
%                         Start here
% --------------------------------------------------------------
 
%\renewcommand{\qedsymbol}{\filledbox}
 
\title{Set Point Weighting for Decoupling}%replace X with the appropriate number
\author{Julius Martensen\\ %replace with your name
Written as a part of the Master Thesis} %if necessary, replace with your course title
 
\maketitle

The Transfer Function of the Closed Loop of a Plant with a Set Point Weighted Controller is given by:

\begin{align*}
y = \left[ I + G~(C_p+C_i)\right]^{-1} ~ G(b~C_p + C_i ) ~r 
\end{align*}

Where $G \in \mathbf{C}^{n \times n}$ is the transfer function of the plant, $C_i \in \mathbf{C}^{n \times n}$ the transfer function of the integral controller, $C_p \in \mathbf{C}^{n \times n}$ the transfer function of the proportional controller, $y \in \mathbf{R}^n$  the output of the plant, $r \in \mathbf{R}^{n \times n}$ and $b \in \mathbf{R}$ the set point weight which describes the influence of the set point on the closed loop. \\

From above, it is clear that the denominator is not influenced by the weight. Hence we can assume a constant characteristic equation of the closed loop.\\

Assuming a first order transfer function for the plant, we can extract the roots of the characteristic equation - see e.g. Aström, Haggalund, Advanced PID p.175 ff - to be:

\begin{align*}
\omega_0 &= ~ \sqrt{ \frac{k_i ~k}{T} }
\end{align*}
Where $\omega_0 \in \mathbf{R}^{+}$ are also called the critical frequency. $k_i \in \mathbf{R}^{+}$ is the integral gain and $k_p \in \mathbf{R}^{+}$ the proportional gain of the controller of the form:

\begin{align*}
C_i &= \frac{1}{T_i~s} ~= ~\frac{k_i}{k_p ~s} \\
C_p &= k_p
\end{align*}
$T \in \mathbf{R}^{+} $ is the Time Constant of the Process Model with a gain $K \in \mathbf{R}^{+}$ given by:

\begin{align*}
G &= ~\frac{K}{T~s+1}
\end{align*}

The proportional gain $k_p$ can be calculated via:

\begin{align*}
k_p &= \frac{2~\zeta ~T ~\omega_0 -1}{K}
\end{align*}

$\zeta \in \left[ 0, 1 \right]$ is the damping of the function. \\

Regarding the Paper by Aström et al. the interaction from input M to output N for a controller of the given form can be written as:

\begin{align*}
|I_N| \leq |k_{NM}||k_{p,M} ~b_M~\omega+k_{i,M}| |M_{s,N}~M_{s,M}|
\end{align*}

Where the interaction $I_N \in \mathbf{R}$ and the Maximum Sensitivity $M_S \in \mathbf{R}$ are given while the interaction indexes $k_{NM} \in \mathbf{R}$ from N to M is given by the Taylor series approximation of the Matrix Q. \\

Assume $I_N, M_{s,N},M_{s,M}$ to be given we can rewrite the equation to be:

\begin{align*}
|k_p~b~\omega + k_i| \leq \underbrace{\frac{|I_N|}{|k_{NM}~M_{s,N}~M_{s_M} |} }_{ \alpha}
\end{align*}


Assuming furthermore the critical frequency to be an upper bound of the inequality and using the already established equations for $k_p$ and $\omega_0$:

\begin{align*}
|\alpha| &\leq | \frac{2~\zeta ~T ~\omega_0 -1}{K} ~b ~  \omega_0  + k_i| \\
& \leq | \frac{2~\zeta ~T ~\sqrt{ \frac{k_i ~k}{T} } -1}{K} ~b ~  \sqrt{ \frac{k_i ~k}{T} } + k_i| \\
& \leq |(1+2 ~\zeta ~b) ~k_i - \sqrt{\frac{b^2}{T~K}}\sqrt{k_i}|
\end{align*}
We can substitute $\sqrt{k_i} = x $ and rewrite:
\begin{align*}
|(1+2~\zeta~b)~x^2 - \sqrt{\frac{b^2}{T~K}} ~x | - |\alpha | \geq 0
\end{align*}
And solve the quadratic equation:
\begin{align*}
x_{1,2} = \frac{\sqrt{\frac{b^2}{T~K}} \pm \sqrt{\frac{b^2}{T~K} + 4 ~ (1+2~\zeta~b) ~\alpha}}{2 ~(1+2~\zeta~b)}
\end{align*}
 
\end{document}