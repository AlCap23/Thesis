\chapter{Preliminaries}
\label{chap:prelim}

The following chapter gives an introduction to the needed theoretical background and the concepts.\\

The section \ref{sec:model} describes the process from a thermo- and fluiddynamic perspective and its implementation in Modelica. First, the thermodynamic process is described and explained in detail. Afterwards, a short introduction regarding Modelica is given. Afterwards the mathematical models describing both machinery and fluids used in the simulations are explained. \\

In section \ref{sec:control} an introduction to the content related to control theory is given. The section starts by explaining the basic concepts of control theory, e.g. feedback control, stability, PID control and robustness. Afterwards, the process model used in this thesis is explained and several possible identification methods are introduced. The section ends with a description on adaptive control.


\section{Process Model}
\label{sec:model}

\section{Control Theory}
\label{sec:control}

The aim of control theory is to influence a given (technical) system $\Sigma$ via a sufficient set of inputs $u \in \mathb{R}^{a}$ in such a way that the systems output $y \in \mathbb{R}^b$ follows a given trajectoy. The behaviour of such a system can be described in different ways. An establish way uses a set of differential equations, given by:

\begin{align}
\label{eq:de}
\Sigma: ~\dot{x} &= f \left(x,u,t\right) \\
y = h\left(x,u,t\right)
\end{align}

Where $t \in \mathbb{R}$ represents the time, $x \in \mathbb{R}^c $ represents the states of a system. Eq.\ref{eq:de} is the general case of a differential equation in time, where the change of the state is given by the function $ f $ and the ouput by the function $h$. This general notion is sufficient for describing almost any system. For the focus of this work, the system will be approximated as a linear, time-invariant (LTI) system. The system is a linear combination with constant coefficients of the time derivatives of both inputs and outputs. Hence, Eq.\ref{eq:de} can be rewritten as

\begin{align}
\label{eq:lti}
\Sigma:~\sum_{i=0}^{n} c_i ~\frac{d^i ~y}{dt^i} &= \sum_{j=0}^{m} d_i ~ \frac{d^j ~u}{dt^j} 
\end{align}

Which we can transform by the La Place Transform into the frequency domain:

\begin{align}
\label{eq:ltis}
\Sigma: ~\sum_{i=0}^{n} c_i ~s^i~Y(s) &= \sum_{j=0}^{m} d_j ~s^j~U(s)  \\
Y(s) &= \frac{\sum_{j=0}^{m} b_j ~s^j}{\sum_{i=0}^{n} a_i ~s^i} U(s) \\
Y(s) &= G(s) ~U(s)
\end{align}

Where $s \in \mathbb{C}$ is the complex frequency. The coefficients have been normalised so that $a_0 = 1$. For the system to be physical  we requiere it to be causal, hence the degree of the denominator must be greater or equal to the the degree of the numerator: $m \leq n$. Hence, the system is only dependent on its previous states and does not depend on its future. The resulting polynomial $G(s) \in \mathbb{C}^{b \times a}$ is called the transfer function. For reasons of simplicity the explicit dependency on the complex frequency will be dropped.\\

In control theory a common concept is called feedback control. The output of the system is measured and compared to a given or wanted value, the set point $Y_{SP} \in \mathbb{C}^b$. While there are other models used more frequently, this work uses a two degree of freedom controller. Both of the signals described are weighted by individual transfer functions. They are called the controller. The setpoint is weighted by the controller $C_{SP} \in \mathbb{C}^{a \times b}$ and the ouput of the system by the controller $C_{Y} \in \mathbb{C}^{a \times b}$. The aim of control theory is design both $C_{SP}$ and $C_{Y}$ in such a way that the closed loop fulfills certain requierements.\\


To understand several of the concepts discussed in this work, a more realistic model pertubed with disturbances $d \in \mathbb{R}^a$ and measurement noise $n \in \mathbb{R}^b$ is presented. The disturbance acts on the systems inputs and hence gives an error in the systems output. The measurement noise acts directly on the measured output of the system. Both signals reduce the truth of the information given to the controller. \\

\textbf{BILD EINFÜGEN} \\

The system can be described by the following equations:

\begin{align}
\label{eq:closedloop}
\end{align}

Introducing the the sensitivity function $S \in \mathbb{C}^{b \times a}$ and the complementary sensitivity function $T$

\begin{align}
\label{eq:sensitivity}
S & = \left( I - G~C_y\right)^{-1} \\
T & = \left( I - G~C_y \right)^{-1} ~G~C_r
\end{align}

Eq. \ref{eq:sensitivity} can be found in Eq. \ref{eq:closedloop}. The sensitivity function describes the influence of measurement noise on the output. The complementary sensitivity function describes the influence of the setpoint on the output. For a single degree of freedom controller they related to each other via the unit matrix, which is given when $G_y ~=~ G_r$. 